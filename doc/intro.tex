Searching for similar sequences is a bioinformatics staple, if not its
actual raison d'\^etre. This is shown by the popularity of programs
for similarity search, above all Blast~\cite{alt90:bas,alt97:gap}, but
also the wide-spread use of a host of other programs for quickly
calculating pairwise local alignments like Blat~\cite{ken02:bla} and
BlastZ~\cite{sch03:hum}. Given this battery of well-established tools,
the problem of finding \emph{dissimilar}, or unique, regions might
seem solved---just carry out a similarity search and take its
complement. In practice, however, there are only few tools available
that allow convenient detection of unique regions in large genomes
like those of mammals, or in large samples of bacerial genomes. And
none of these tools is as established as one of the popular aligners.

This repository accompanies a forthcoming review on searching for
unique regions in genome sequences~\cite{vie25:fas}. We survey three
published tools for finding unique
regions, \ty{genmap}~\cite{poc20:gen}, \ty{macle}~\cite{pir19:hig}
and \ty{fur}~\cite{hau21:fur,vie24:mar}, by applying them to simulated
and real data. We first analyze single query sequences before
comparing pairs of query and subject sequences. The query/subject
nomenclature is well known from Blast and its variants. However, it
turns out that when searching for unique regions, the most interesting
combination of query/subject sequences are subjects that are the
closest distinct relatives of the queries. To emphasize this, we have
called the queries ``targets'' and the subjects ``neighbors'' in the
review and also later in this document, starting in Chapter 3.

As a last preliminary, we occasionally quote measurements of resource
consumption. These refer to runs under Ubuntu 24.04 on WSL for
Windows11 on a consumer-grade laptop with 16 12th Gen Intel i7-1260P
CPUs and 32\,GB of RAM.
